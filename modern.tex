\section{Moderna kryptosystem}
Moderna kryptosystem är helt och hållet baserade på matematik, exempelvis 
resultat inom talteori och abstrakt algebra.
De används dessutom till fler saker än att bara hålla information hemlig.
Dagens kryptografi handlar också om att informationen ska kunna verifieras, för 
att se att ingen har ändrat på ett meddelande, och att se om det är rätt 
avsändare av meddelandet.
Denna typ av kryptografi kallas \emph{public key cryptography}
\index{public-key cryptography|see{asymmetrisk kryptering}} eller 
\emph{asymmetrisk kryptering}\index{asymmetrisk 
kryptering}\index{kryptering!asymmetrisk}.
Alla chiffer som diskuterats i föregående avsnitt är av typen \emph{symmetrisk 
kryptering}\index{symmetrisk kryptering}\index{kryptering!symmetrisk} där samma 
nyckel används för både kryptering och avkryptering.
I asymmetrisk kryptering används alltså olika nycklar för kryptering och 
avkryptering.

Mycket av dagens kryptografi används i mobiltelefoner och datorer.
Samtalet är krypterat från mobiltelefonen till basstationen, det vill säga 
under den sträcka det färdas genom luften som radiovågor.
Anslutningen till en webbserver är krypterad när inloggningsuppgifter skickas 
till servern, exempelvis när man loggar in till sitt e-postkonto.
Kryptografi används även för att verifiera att det är rätt webbserver som man 
kommunicerar med, för att undvika att skicka uppgifter till någon som låtsas 
vara rätt server.
Det är därför viktigt att se i webbläsaren så att det inte är en falsk server 
som man anslutit till.
% XXX referenser för attacker mot YouTube, Facebook och Google.
Detta visas i webbläsaren på olika otydliga vis, beroende på webbläsare, men de 
har blivit tydligare de senaste åren eftersom att antalet attacker mot populära 
sajter som Facebook, YouTube och Google också ökat.
Anledningarna till en sådan attack kan vara olika, från en regering som vill 
kontrollera sina invånare till kriminella organisationer som antingen vill lura 
åt sig pengar eller sälja uppgifterna till någon som vill använda dem.

Kryptografi är alltså en viktig del av den tekniska vardagen, men sker oftast 
utan att vi märker av den.

För en vidare diskussion om moderna chiffer se Stinsons bok \emph{Cryptography: 
Theory and practice}~\cite{Stinson2006cta}, och för en mer översiktlig bild 
tillsammans med andra aspekter på säkerhet se Andersons bok \emph{Security 
Engineering}~\cite{Anderson2008sea}.

