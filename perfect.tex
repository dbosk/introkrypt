\section{Engångschiffer och perfekt sekretess}
Vi inleder detta avsnitt med att definiera vad vi menar med perfekt 
sekretess\footnote{%
  Engelskans \emph{perfect secrecy}.
}.
Detta begrepp publicerades första gången av Shannon~\cite{Shannon1949cto} år 
1949.
\begin{definition}\label{def:perfectSecrecy}\index{perfekt sekretess!formell 
    definition}
  Ett kryptosystem \((\P, \C, \K, \E, \D)\) sägs ha \emph{perfekt sekretess} om 
  \(\Pr(\p = p\mid \c = c) = \Pr(\p = p)\) för alla \(p\in \P\) och \(c\in 
  \C\).
  Det vill säga, sannolikheten a posteriori att en klartext är \(p\) om 
  kryptotexten är \(c\) är densamma som sannolikheten a priori att klartexten 
  är \(p\).
\end{definition}

Låt oss fortsätta med att visa några resultat om perfekt sekretess.
Vi inleder med följande lemma som fastställer för en typ av kryptosystem några 
egenskaper som krävs för att detta system ska tillhandahålla perfekt sekretess.
\begin{lemma}\label{lem:perfectSecrecy}
  Låt \((\P,\C,\K,\E,\D)\) vara ett kryptosystem.
  Om \(|\K| = |\C| = |\P|\) och varje nyckel används med sannolikheten 
  \(1/|\K|\) och det för varje klartext \(p\in \P\) och kryptotext \(c\in \C\) 
  finns en unik nyckel \(k\in \K\) sådan att \(e_k(p) = c\), då tillhandahåller 
  kryptosystemet perfekt sekretess.
\end{lemma}
\begin{proof}
  Antag \(|\K| = |\C| = |\P|\).
  Vidare antag \(\Pr(\k = k) = 1/|\K|\) för alla nycklar \(k\in \K\) och att 
  det för alla klartexter \(p\in \P\) och kryptotexter \(c\in \C\) finns en 
  nyckel \(k\in \K\) sådan att \(e_k(p) = c\) och för alla nycklar 
  \(k^\prime\neq k\) gäller att \(e_{k^\prime}(x)\neq c\).

  Låt \(c\in \C\) vara en godtycklig kryptotext.
  Då har vi att
  \begin{align}
    \nonumber
    \Pr(\c = c) = \sum_{k\in \K}\Pr(\k = k)\Pr(\p = d_k(c)).
  \end{align}
  Eftersom att \(\Pr(\k = k) = 1/|\K|\) för alla möjliga \(k\in \K\) och att 
  nyckeln är unik för varje klartext--kryptotextpar har vi
  \begin{align}
    \nonumber
    \Pr(\c = c) &= \sum_{k\in \K}\frac{1}{|\K|}\Pr(\p = d_k(c)) \\
    \nonumber
      &= \frac{1}{|\K|}\sum_{k\in \K} \Pr(\p = d_k(c)).
  \end{align}
  För en fixerad kryptotext \(c\in \C\) är \(d_k(c)\) en permutation av \(\P\).
  Följaktligen får vi
  \begin{align}
    \nonumber
    \Pr(\c = c) &= \frac{1}{|\K|}\sum_{p\in \P} \Pr(\p = p) \\
    \nonumber
      &= \frac{1}{|\K|}\times 1 = \frac{1}{|\K|}.
  \end{align}

  Vidare har vi att \(\Pr(\c = c\mid \p = p) = \Pr(\k = k)\) tack vare att det 
  är en unik nyckel \(k\in \K\) för varje par av klartext \(p\in \P\) och 
  kryptotext \(c\in \C\).

  % XXX cross-reference Bayes' theorem, \cref{thm:Bayes}
  Slutligen får vi då genom Bayes sats att
  \begin{align}
    \nonumber
    \Pr(\p = p\mid \c = c) &= \frac{\Pr(\p = p)\Pr(\c = c\mid \p = p)}{\Pr(\c 
    = c)} \\
    \nonumber
      &= \frac{\Pr(\p = p)\frac{1}{|\K|}}{\frac{1}{|\K|}} = \Pr(\p = p).
  \end{align}
  Då \(\Pr(\p = p\mid \c = c) = \Pr(\p = p)\) har vi perfekt sekretess.
\end{proof}

Vi fortsätter med ett mer generellt resultat som visar att kryptosystem av 
denna typ som uppfyller perfekt sekretess måste uppfylla dessa egenskaper.
\begin{theorem}[Shannons sats]\label{thm:perfectSecrecy}\index{Shannons 
    sats}\index{perfekt sekretess!Shannons sats}
  Antag att \((\P, \C, \K, \E, \D)\) är ett kryptosystem sådant att \(|\K| 
  = |\C| = |\P|\).
  Detta kryptosystem tillhandahåller perfekt sekretess om och endast om
  varje nyckel \(k\in \K\) används med lika sannolikhet \(1/|\K|\) och det för 
  varje klartext \(p\in \P\) och kryptotext \(c\in \C\) finns en unik nyckel 
  \(k\in \K\) sådan att \(e_k(p) = c\).
\end{theorem}
\begin{proof}
  Vi har redan enligt \cref{lem:perfectSecrecy} att ett kryptosystem med 
  dessa egenskaper ger perfekt sekretess.
  Det som återstår att visa är att ett system som uppfyller perfekt sekretess 
  måste vara ett sådant system.

  Låt oss därför anta \(|\K| = |\C| = |\P|\) och att detta kryptosystem ger 
  perfekt sekretess, det vill säga \(\Pr(\p = p\mid \c = c) = \Pr(\p = p)\).
  Från definitionen av kryptosystem (\cref{def:kryptosystem}) har vi att 
  för alla klartexter \(p\in \P\) och kryptotexter \(c\in \C\) existerar 
  åtminstone en nyckel \(k\in \K\) sådan att \(e_k(p) = c\), det vill säga
  \begin{align}
    \nonumber
    |\C| = |\{e_k(p)\colon k\in \K\}|\leq |\K|.
  \end{align}
  Men enligt vårt antagande om typen av system är \(|\C| = |\K|\), alltså kan 
  det inte finnas två nycklar \(k\in \K\) och \(k^\prime\in \K\) sådana att 
  \(k\neq k^\prime\) och \(e_k(p) = c\).

  Vidare fixera en kryptotext \(c\in \C\), låt \(\P = \{p_i\colon 1\leq i\leq 
  n\}\) där \(n = |\K| = |\P|\) och indexera nycklarna \(k_i\in \K\) sådana att 
  \(e_{k_i}(p_i) = c\), för \(1\leq i\leq n\).
  % XXX cross-reference Bayes' theorem, \cref{thm:Bayes}
  Genom Bayes sats har vi då
  \begin{align}
    \nonumber
    \Pr(\p = p_i\mid \c = c) &= \frac{\Pr(\p = p_i)\Pr(\c = c\mid \p 
    = p_i)}{\Pr(\c = c)} \\
    \nonumber
      &= \frac{\Pr(\p = p_i)\Pr(\k = k_i)}{\Pr(\c = c)}.
  \end{align}
  Eftersom att vi har perfekt sekretess \(\Pr(\p = p_i\mid \c = c)\) får vi att
  \begin{align}
    \nonumber
    \frac{\Pr(\p = p_i)\Pr(\k = k_i)}{\Pr(\c = c)} = \Pr(\p = p_i).
  \end{align}
  och således att \(\Pr(\k = k_i) = \Pr(\c = c)\).
  Då vi valt ett godtyckligt \(c\) måste \(\k\) ha ett likformigt 
  sannolikhetsmått.
  Följaktligen måste \(\Pr(\k = k_i) = 1/|\K|\) för alla nyklar \(k_i\in \K\).
\end{proof}

\index{perfekt sekretess}
Det \cref{thm:perfectSecrecy} säger är att om vi använder ett 
Vigenèrechiffer, eller motsvarande kryptosystem, med en nyckel som är lika lång 
som klartexten och aldrig någonsin återanvänder nyckeln, då kommer kryptotexten 
att vara oknäckbar.
Låt oss även ge en mer intuitiv förklaring.
Säg att vi har krypterat klartexten \(p\in \P\) med nyckeln \(k\in \K\) och 
fått kryptotexten \(c\in \C\).
Angriparen kan då för varje möjlig klartext \(p^\prime\in \P\) hitta en nyckel 
\(k^\prime\in \K\) sådan att \(e_{k^\prime}(p^\prime) = c\).
Det kommer följaktligen vara omöjligt att avgöra om \(p^\prime\) eller \(p\) är 
den riktiga klartexten utan att ha mer information, båda klartexterna kommer 
att ha lika sannolikhet.

I \cref{ex:VigenereKolumner} kunde vi knäcka chiffret eftersom att 
nyckellängden var fyra medan längden av klartexten var sju gånger längre.
Antalet möjliga nycklar \(|\K|\) var alltså inte detsamma som antalet möjliga 
klartexter \(|\P|\), följaktligen gick det enligt 
\cref{thm:perfectSecrecy} ej att uppnå perfekt sekretess i det fallet.

\begin{exercise}
  Formulera en sats med bevis som bestämmer vad gäller perfekt sekretess för 
  substitutionschiffer (\cref{def:substitutionCipher}), där \(|\P| 
  = |\C|\neq |\K|\).
\end{exercise}
\begin{exercise}
  Detsamma gäller permutationschiffer (\cref{def:permutationCipher}), 
  formulera en sats med bevis gällandes perfekt sekretess för detta chiffer.
\end{exercise}
\begin{exercise}
  Går det att dra någon generell slutsats vad gäller perfekt sekretess för 
  kryptosystem där \(|\P| = |\C|\neq |\K|\)?
  Bevisa denna slutsats eller visa att ingen sådan kan dras.
\end{exercise}

\subsection{Vernams engångschiffer}
\label{sec:OTP}
\index{Vernams engångschiffer|see{engångschiffer}}\index{engångschiffer}
Faktum är att redan år 1917 hade Gilbert Vernam beskrivit ett chiffer med 
egenskaperna som krävs i \cref{thm:perfectSecrecy}~\cite{Stinson2006cta}, 
det vill säga långt innan Shannon hade publicerat teorin för att matematiskt 
visa perfekt sekretess.
Detta engångschiffer, mer känt som \emph{One-time Pad} (OTP), ges i följande 
definition.
\begin{definition}[One-time Pad]\index{engångschiffer!formell 
    definition|see{one-time pad}}\index{one-time pad}
  Låt \(n\) vara ett positivt heltal.
  Definiera att \(\P = \C = \K = (\Z_2)^n\).
  För alla nycklar \(k = (k_1,\ldots,k_n)\in \K\), klartexter \(p = (p_1, 
  \ldots, p_n)\in \P\) och kryptotexter \(c = (c_1,\ldots,c_n)\in \C\) 
  definierar vi att
  \begin{align}
    \nonumber
    e_k(p) &= (p_1 + k_1, \ldots, p_n + k_n),
  \end{align}
  där alla operationer utförs i \(\Z_2\), och därefter definierar vi att \(d_k 
  = e_k\).

  Nyckeln \(k\in \K\) måste väljas slumpmässigt och får aldrig återanvändas.
\end{definition}

Detta är ekvivalent med att arbeta med \(n\) bitar långa bitsträngar och där 
\(e_k(p) = p\xor k\) och \(d_k(c) = c\xor k\), den binära operationen \(\xor\) 
är \emph{bitvis exklusivt eller} (XOR).

\begin{exercise}
  Låt \(p, p^\prime\in \P\) vara två klartexter och låt \(k\in \K\) vara en 
  kryptonyckel för OTP.\@
  Om vi krypterar de båda klartexterna med samma nyckel, \(e_k(p)\) och 
  \(e_k(p^\prime)\), visa en attack som tar bort beroendet av nyckeln.
\end{exercise}

