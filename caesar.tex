\section{Caesarchiffer}
\label{sec:Caesar}
\index{Caesarchiffer}
Chiffret vi ska titta på i detta avsnitt är uppkallat efter den romerske 
diktatorn och kejsaren Julius Caesar (49 f.Kr. -- 44 e.Kr.).
Även om chiffret troligtvis uppfunnits tidigare har det fått detta namn 
eftersom att Caesar lär ha använt det med nyckeln given 
i \cref{tbl:Caesarchiffer}~\cite{Stinson2006cta}.
Chiffret är annars även känt som ett skiftchiffer, vi kommer att se varför.

Chiffret använder det vanliga alfabetet som både klartext- och kryptoalfabete.
För att kryptera förskjuts kryptoalfabetet mot klartextalfabetet ett givet 
antal steg.
Det är antalet steg som utgör nyckeln i Caesarchiffret.
Därefter krypteras meddelandet genom att varje klartextbokstav motsvaras av en 
kryptotextbokstav.
Se \cref{tbl:Caesarchiffer}.

\begin{table*}
  \caption{%
    Tabell för att kryptera med ett Caesarchiffer med nyckeln C.
  }\label{tbl:Caesarchiffer}
	\centering
  \begin{tabular}{ccccccccccccccc}
    \toprule
    a & b & c & d & e & f & g & h & i & j & k & l & m & n & o \\
		C & D & E & F & G & H & I & J & K & L & M & N & O & P & Q \\
    \midrule
    p & q & r & s & t & u & v & w & x & y & z & å & ä & ö \\
		R & S & T & U & V & W & X & Y & Z & Å & Ä & Ö & A & B \\
    \bottomrule
  \end{tabular}
\end{table*}

\begin{example}
  För att kryptera klartexten \emph{hej} slår man upp bokstav för bokstav 
  i \cref{tbl:Caesarchiffer}.
	Det vill säga, \emph{h $\mapsto$ J}, \emph{e $\mapsto$ G} och 
	\emph{j $\mapsto$ L}.
	Kryptotexten blir alltså \emph{JGL}.
\end{example}

\begin{example}\label{ex:CaesarSkatten}
	Om vi krypterar ordet \emph{skatten} blir det \emph{UMCVVGP}.
\end{example}

\subsection{Formell definition av Caesarchiffret}
Låt oss ge följande definition av Caesar- eller skiftchiffret.
\begin{definition}[Skiftchiffer]\label{def:shiftCipher}\index{Caesarchiffer!formell 
    definition}\index{skiftchiffer!formell definition}
  Låt \(\P = \C = \K = \Z_{29}\) och låt varje bokstav i det svenska alfabetet 
  motsvara ett unikt tal i \(\Z_{29}\).
  För alla \(k\in \K\) definierar vi
  \begin{align}
    \nonumber
    e_k(p) &= (p + k) \bmod 29, \text{\ och\ } \\
    \nonumber
    d_k(c) &= (c - k) \bmod 29,
  \end{align}
  där \(p\in \P\) är en klartextbokstav och \(c = e_k(p)\in \C\) är motsvarande 
  kryptotextbokstav.
\end{definition}

Vi förtydligar definitionen med ett exempel.
\begin{example}\label{ex:shiftdef}
  Låt oss numrera bokstäverna i det svenska alfabetet enligt index med start 
  från noll.
  Då får vi att textsträngen \enquote{hej} skulle kunna motsvaras av tupeln \(p 
    = (7, 4, 9)\).
  Om vi låter nyckeln \(k\in \K\) vara \(2\) får vi att
  \begin{align}
    \nonumber
    c = e_2(p) &= (e_2(7), e_2(4), e_2(9)) \\
    \nonumber
      &= (9, 6, 11).
  \end{align}
  Om vi översätter tillbaka till bokstäver får vi att \(c\) motsvarar strängen 
  \enquote{JGL}.
\end{example}

Vi ser här att antalet möjliga nycklar \(|\K| = |\Z_{29}| = 29\) är alldeles 
för få.

\begin{exercise}
  Försök att finna en kryptotext \(c\) och två nycklar, \(k\) och \(k^\prime\), 
  sådana att \(d_k(c)\) och \(d_{k^\prime}(c)\) ger tolkningsbara klartexter då 
  Caesarchiffret används som kryptosystem.
\end{exercise}
\begin{exercise}
  Hur påverkar längden av kryptotexten i förhållandet till kryptosystemets 
  blockstorlek, i Caesarchiffrets fall är blockstorleken en bokstav?
\end{exercise}

\subsection{Kryptanalys av Caesarchiffret}
\label{sec:KryptanalysCaesar}
\index{Caesarchiffer!kryptanalys}
Caesarchiffret är inte ett särskilt säkert sätt att skydda information.
Det är lätt att knäcka.
Det finns totalt, om det svenska alfabetet används, 29 olika nycklar som kan 
användas för kryptering och avkryptering eftersom att alfabetet maximalt kan 
förskjutas lika många steg som det finns bokstäver\footnote{%
	Detta kan beräknas genom att vi på den första platsen kan välja mellan 29 
	bokstäver, på de efterföljande platserna kan då bara välja en bokstav.  Vi 
	får då totala antalet nycklar genom \(29\cdot 1\cdot 1\cdots 1 = 29\).
}.
Detta är så få att det till och med enkelt kan testas för hand för att lista ut 
vilken nyckel som använts.
Om det finns tillgång till en dator och man kan programmera, då är det ännu 
enklare.
Men det går tack vare språkets egenskaper att reducera antalet nycklar som 
behöver testas ytterligare.
Titta på \cref{ex:CaesarSkatten} där \emph{tt} blir \emph{VV}, det är 
långt från alla bokstäver i svenskan som upprepas på detta sätt.
I \cref{sec:KryptanalysSubstitution} ska vi se ytterligare ett sätt att 
kryptanalysera Caesarchiffret på.

\begin{exercise}\label{xrc:caesarAddition}
  Låt \(c\) och \(c^\prime\) vara två kryptotexter krypterade med samma nyckel 
  \(k\).
  Det vill säga \(c = e_k(m) = (m_1+k, m_2+k, \ldots, m_n+k)\) och \(c^\prime 
  = e_k(m^\prime) = (m^\prime_1+k, m^\prime_2+k, \ldots, m^\prime_n+k)\), där 
  alla aritmetiska operationer görs modulo 29.
  Undersök vad som händer då vi utför aritmetik med \(c\) och \(c^\prime\), 
  exempelvis \(c - c^\prime \mod 29\).
  Hur påverkar detta inflytandet av nyckeln \(k\) på kryptotexten?
\end{exercise}

