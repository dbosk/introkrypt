\section{Vigenèrechiffer}
\index{Vigenèrechiffer}
\citet{Singh2000tcb} har i sin bok \emph{Kodboken} gjort en kartläggning över 
uppkomsten och förekomster av olika chiffer under historiens gång.
Enligt honom lades grunden för Vigenèrechiffret under 1400-talet av Leon 
Battista Alberti (1404--1472).
Därefter vidareutvecklades hans idéer först av Johannes Trithemius (1462--1516) 
och sedan av Giambattista della Porta (1535--1615).
Anledningen till att metoden kallas Vigenèrechiffer är enligt honom för att den 
är uppkallad efter Blaise de Vigenère (1523--1596) som gjorde det slutgiltiga 
bidraget till utformningen av chiffret.
Vigenèrechiffret användes länge, det användes till och med av sydstaterna under 
det amerikanska inbördeskriget.

Chiffret består av upprepad användning av Caesarchiffret.
Som nyckel används ett ord, för att vara enkelt att komma ihåg, vilket 
bokstavskombination som helst kan användas.
Vid kryptering av en text krypteras första bokstaven i klartexten med ett 
Caesarchiffer där första bokstaven i Vigenèrenyckeln används som nyckel.
Därefter används den andra, den tredje, och så vidare.
När nyckelordets alla bokstäver använts börjar man om.

\begin{example}\label{ex:VigenereSkatten}
  Om vi vill kryptera order \emph{skatten} ska bokstäverna i nyckeln användas 
  enligt
  \begin{verbatim}
    skatten
    ABCABCA
  \end{verbatim}
  och vi får alltså \emph{SLCTUGN} genom att använda de olika Caesarchiffren 
  i \cref{tbl:Vigenerechiffer}.
\end{example}

\begin{table*}
  \caption{%
    Vigenèrechiffer med nyckeln \emph{ABC}.
  }\label{tbl:Vigenerechiffer}
  \centering
  \begin{tabular}{lcccccccccc}
    \toprule
    Klartext & a & b & c & d & e & f & g & h & i & j \\
    A        & A & B & C & D & E & F & G & H & I & J \\
    B        & B & C & D & E & F & G & H & I & J & K \\
    C        & C & D & E & F & G & H & I & J & K & L \\
    \midrule
    Klartext & k & l & m & n & o & p & q & r & s & t \\
    A        & K & L & M & N & O & P & Q & R & S & T \\
    B        & L & M & N & O & P & Q & R & S & T & U \\
    C        & M & N & O & P & Q & R & S & T & U & V \\
    \midrule
    Klartext & u & v & w & x & y & z & å & ä & ö \\
    A        & U & V & W & X & Y & Z & Å & Ä & Ö \\
    B        & V & W & X & Y & Z & Å & Ä & Ö & A \\
    C        & W & X & Y & Z & Å & Ä & Ö & A & B \\
    \bottomrule
  \end{tabular}
\end{table*}

Notera skillnaden mellan kryptotexten av ordet \emph{skatten} i 
\cref{ex:CaesarSkatten}, \cref{ex:SubstitutionSkatten} och 
\cref{ex:VigenereSkatten}.
Upprepningen av t:et försvinner när Vigenerechiffret används.

\subsection{Formell definition av Vigenèrechiffret}
Vi går vidare med en definition av Vigenèrechiffret.
\begin{definition}[Vigenèrechiffer]\index{Vigenèrechiffer!formell definition}
  Låt \(n\) vara ett positivt heltal.
  Definiera att \(\P = \C = \K = (\Z_{29})^n\).
  För alla nycklar \(k = (k_1,\ldots,k_n)\in \K\), klartexter \(p = (p_1, 
  \ldots, p_n)\in \P\) och kryptotexter \(c = (c_1,\ldots,c_n)\in \C\) 
  definierar vi att
  \begin{align}
    \nonumber
    e_k(p) &= (p_1 + k_1, \ldots, p_n + k_n), \text{\ och\ } \\
    \nonumber
    d_k(c) &= (c_1 - k_1, \ldots, c_n - k_n),
  \end{align}
  där alla operationer utförs i \(\Z_{29}\).
\end{definition}

Vi noterar att den enda skillnaden mellan denna definition och 
\cref{def:shiftCipher} är att \(\P, \C, \K\) definieras som 
\((\Z_{29})^n\) istället för \(\Z_{29}\).
Låter vi \(n = 1\) är dessa system identiska.

Om vi använder gruppen \(\Z_2\) istället för \(\Z_{29}\) kommer vi att arbeta 
med bitsträngar av längden \(n\) bitar.
Detta får effekten att \(e_k(p) = p \xor k\) där \(p\) och \(k\) är bitstängar 
av längd \(n\), det vill säga operationen \emph{bitvis exklusivt eller} (XOR).
Detta är en fundamental operation i dagens datorer.

\begin{exercise}
  Försök att finna en kryptotext \(c\) som är lika lång som nyckeln och två 
  nycklar, \(k\) och \(k^\prime\), sådana att \(d_k(c)\) och 
  \(d_{k^\prime}(c)\) ger tolkningsbara klartexter med detta kryptosystem.
\end{exercise}
\begin{exercise}
  Om vi ändrar villkoret i föregående övning, vad händer då kryptotexten 
  istället är längre än nyckeln?
  Och mer intressant, varför blir det så?
\end{exercise}

\subsection{Kryptanalys av Vigenèrechiffret}
\index{Vigenèrechiffer!kryptanalys}
Eftersom att kryptotexten nu är krypterad med flera Caesarnycklar fungerar inte 
längre metoden som vi tog fram i \cref{sec:KryptanalysSubstitution}.
Friedrich Kasiski (1805--1881) publicerade år 1863 tekniken hur man 
fullständigt knäcker chiffret utan några förkunskaper~\cite{Stinson2006cta}.
Tidigare metoder, före Kasiski, krävde att man kände till delar av klartexten,
att man kunde gissa nyckeln eller kände nyckelns längd.

Med mycket kryptotext är det möjligt att finna upprepningar i kryptotexten.
Avståndet mellan upprepningarna måste vara en multipel av nyckelns längd 
eftersom att samma klartext annars skulle krypteras olika på grund av att olika 
delar av nyckeln används.
Det vill säga, nyckelns längd måste vara en gemensam faktor för alla avstånd 
mellan upprepningar.
Om vi tittar på följande exempel.
\begin{example}\label{ex:VigenereMedUpprepning}
  Ett Vigenèrechiffer med nyckeln \emph{ABCD} används för att kryptera texten 
  \emph{cryptoisshortforcryptography}.
  \begin{verbatim}
    Nyckel:     ABCDABCDABCDABCDABCDABCDABCD
    Klartext:   cryptoisshortforcryptography
    Kryptotext: CSASTPKVSIQUTGQUCSASTPIUAQJB
  \end{verbatim}
  Avståndet mellan den upprepade texten \emph{CSASTP} är 16, från första 
  tecken till första tecken.
  De möjliga nyckellängderna är alltså \(16, 8, 4, 2\) eller \(1\).
\end{example}
Genom att finna flera sådana upprepningar är det möjligt att reducera antalet 
möjliga nyckellängder.

När nyckellängden väl är känd, låt oss säga att den är \(n\) tecken, då skrivs 
kryptotexten med \(n\) teckens bredd.
Som vi ser i \cref{ex:VigenereMedUpprepning} hamnar då alla tecken 
krypterade med samma Caesarnyckel ovanför varandra i en kolumn, se 
\cref{ex:VigenereKolumner}.
\begin{example}\label{ex:VigenereKolumner}
  Ett Vigenèrechiffer med nyckeln \emph{ABCD} används för att kryptera texten 
  \emph{cryptoisshortforcryptography}.
  \begin{verbatim}
    Nyckel:     ABCD
    Klartext:   cryp
                tois
                shor
                tfor
                cryp
                togr
                aphy
    Kryptotext: CSAS
                TPKV
                SIQU
                TGQU
                CSAS
                TPIU
                AQJB
  \end{verbatim}
\end{example}
Eftersom att varje kolumn nu är krypterad endast med ett Caesarchiffer kan vi 
enkelt använda kryptanalysmetoderna från \cref{sec:KryptanalysCaesar}
eller \cref{sec:KryptanalysSubstitution} för att lista ut varje 
Caesarnyckel och därmed hela Vigenèrenyckeln.
I \cref{ex:VigenereKolumner} analyserar vi den första kolumnen för att 
komma fram till att den är krypterad med nyckeln \emph{A}, den andra kolumnen 
är krypterad med nyckeln \emph{B}, och så vidare, och slutligen att 
Vigenèrechiffrets nyckel är \emph{ABCD}.

\begin{exercise}
  Undersök resultatet i \cref{xrc:caesarAddition}, hur tillämpbart är 
  detta för Vigenèrechiffret?
\end{exercise}
