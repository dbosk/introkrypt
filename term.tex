\section{Terminologi för kryptosystem}
\noindent
När vi pratar om kryptografi används viss terminologi.
Vi har en klartext och ett klartextalfabet.
\emph{Klartexten}\footnote{%
	Engelskans \emph{plaintext}.
}\index{klartext} är det hemliga meddelande som vi vill skydda med hjälp av 
kryptografi.
\emph{Klartextalfabetet}\footnote{%
	Engelskans \emph{plaintext alphabet}.
}\index{klartextalfabet} är det alfabet som används för att skriva klartexten.

Sedan har vi också en kryptotext och ett kryptoalfabet.
\emph{Kryptotext}\footnote{%
	Engelskans \emph{ciphertext}.
}\index{kryptotext} är den resulterande texten som vi får efter att vi 
krypterat vår klartext.
\emph{Kryptoalfabetet}\footnote{%
	Engelskans \emph{ciphertext alphabet}.
}\index{kryptoalfabet} är det alfabet som används för kryptotexten.

I de kryptosystem som finns i denna text används olika delar av det vanliga 
alfabetet som klartextalfabet respektive kryptoalfabet.
För att kunna skilja på vilket som är vilket väljer vi våra gemener för 
klartextalfabetet, exempelvis \emph{abc\dots}, och våra versaler för 
kryptoalfabetet, exempelvis \emph{ABC\dots}.

För att kunna kryptera och avkryptera krävs en \emph{hemlig nyckel}\footnote{%
	Engelskans \emph{secret key}.
}\index{hemlig nyckel}, det är alltså nyckeln som ska hållas hemlig.
För att kunna avkryptera ett hemligt meddelande, en kryptotext, krävs nyckeln.
Med fel nyckel ger avkrypteringen bara en text med osammanhängande 
kombinationer av tecken från klartextalfabetet.

\subsection{Formell definition av ett kryptosystem}
\noindent
Låt oss inleda med att definiera vad vi menar när vi skriver kryptosystem.
Vi kommer i denna text att använda samma matematiska notation som 
\citet{Stinson2006cta}.
\begin{definition}\label{def:kryptosystem}
  Ett \emph{kryptosystem}\index{kryptosystem!formell definition} är en tupel 
  \((\P, \C, \K, \E, \D)\) där följande gäller:
  \begin{enumerate}
    \item \(\P\) är en ändlig mängd av möjliga klartexter.
    \item \(\C\) är en ändlig mängd av möjliga kryptotexter.
    \item \(\K\), kallad \emph{nyckelrymden}, är en ändlig mängd av möjliga 
      nycklar.
    \item För varje \(k\in \K\) finns en 
      \emph{krypteringsregel}\index{krypteringsregel} \(e_k\in \E\) och 
      motsvarande \emph{avkrypteringsregel}\index{avkrypteringsregel} \(d_k\in 
      \D\).
      Varje \(e_k\colon \P\to \C\) och \(d_k\colon \C\to \P\) är funktioner 
      sådana att \(d_k(e_k(p)) = p\) för alla klartexter \(p\in \P\).
  \end{enumerate}
\end{definition}
Det är den sistnämda egenskapen som gör att vi kan kommunicera utan 
tvetydigheter.
Samma egenskap säger också att det är nyckeln \(k\) som måste hållas hemlig för 
att vår kommunikation ska hållas säker.
